\chapter{Lecture 12/13: Simple Models of Walking and Running}
In this chapter we'll introduce some of the simple models of walking and robots, the control problems that result, and a very brief summary of some of the control solutions described in the literature. Compared to the robots that we have studied so far, our investigations of legged locomotion will require additional tools for thinking about limit cycle dynamics and dealing with impacts.


\section{Limit Cycles}
In many of the systems that we have studied so far, we have analyzed the stability of a fixed-point, or even an (infinite-horizon) trajectory. For walking systems the natural equivalent is to talk about the stability of periodic solutions -- a fixed "gait" is a cycle that repeats footstep after footstep. So we begin our discussion with a discussion of the stability of a cycle.
A limit cycle is an asymptotically stable or unstable periodic orbit. One of the simplest models of limit cycle behavior is the Van der Pol oscillator.
\subsection{Poincare Maps}


\section{Simple Models of Walking}
\subsection{The Rimless Wheel}
The most elementary model of passive dynamic walking, first used in the context of walking by, is the rimless wheel. This simplified system has rigid legs and only a point mass at the hip as illustrated in the figure above. To further simplify the analysis, we make the following modeling assumptions
\begin{itemize}
\item No slip
\item Collisions are inelastic and impulsive (no bouncing)
\item No double support
\end{itemize}
\subsection{The Compass Gait}
\subsection{The Kneed Walker}


\section{Simple Models of Running}
There are existing various definitions of running:
\begin{itemize}
\item Existence of an Aerial Phase
\item Exchange of Energy 
\end{itemize}
Why do we study simple models?
\begin{itemize}
\item Tractable
\item Mechanical Insights
\item Comparative Biology: Fundamental Principles?
\item As a 'Template' for Higher-DOF Robots
\end{itemize}
\subsection{The Spring-Loaded Inverted Pendulum}
\subsubsection{Assumptions}
\begin{itemize}
\item Masless leg -> command $\theta$ instantaneously
\item Perfectly \textbf{elastic} collision -> Energy is always conserved (thread to stability)
\item When the foot is on ground, we have a pin joint i.e. infinite friction (no sliding)
\end{itemize}
\subsubsection{SLIP Modeling}
The model is a point mass, $m$ , on top of a massless, springy leg with rest length of
$l_0$, and spring constant $k$. The state of the system is given by $x,y$ the position of the center of mass, and the length,$l$ , and angle $\theta$ of the leg. Like the rimless wheel, the dynamics are modeled piecewise - with one dynamics governing the flight phase, and another governing the stance phase.

Flight Phase. State variables: $\myM{x}=[x,y,\dot{x},\dot{y}]^T$. Dynamics are
$$\dot{\myM{x}}\begin{bmatrix} \dot{x} \\ \dot{y} \\ 0 \\ - g
      \end{bmatrix}.$$
      
Stance Phase. State variables: $\myM{x}=[r,\theta, \dot{r},\dot{\theta}]$. Kinematics are
$$x = \begin{bmatrix} - r \sin\theta \\ r
      \cos\theta \end{bmatrix}.$$
Energy is given by
$$T = \frac{m}{2}
      (\dot{r}^2 + r^2 \dot\theta^2 ), \quad U = mgr\cos\theta +
      \frac{k}{2}(r_0 - r)^2.$$
Putting these into Lagrange yields:
\begin{equation}
  m \ddot{r} - m r \dot\theta^2 + m g \cos\theta - k (r_0 -r) = 0  
\end{equation}
\begin{equation}
m r^2 \ddot{\theta} + 2mr\dot{r}\dot\theta - mgr \sin\theta = 0
\end{equation}
\subsubsection{SLIP Control}
Choose $\theta_{touchdown}$ during aerial phase. 

Goal: Design controller $u[n]=\Pi(y[n])$ to stabilize $y^d$.
\begin{itemize}
\item Idea 1: find $u\star$ st. $y^\star=P(y^d,u^\star)$. 

\textbf{Linearize} P around $(y^d, u^p)$ + do (discrete time) \textbf{LQR}.

Results in an exponential convergence to the fixed point. 
\item Idea 2: Deadbeat Control.

If P is invertible,
$$u[n]=P^{-1}(y^d,y[n])$$
If exists, results in an convergence to the fixed-point via one timestep.
\end{itemize}

\subsection{Continuous Control: The Planar Monopod Hopper}
\begin{itemize}
\item Hopping Height (push at toe-off)
\item Foot Touchdown to regulate speed
\item Stabilize attitude during stance
\end{itemize}

\section{A Simple Model That Can Walk and Run}







