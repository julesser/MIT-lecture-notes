\chapter{Lecture 16: Highly-Articulated Legged Robots, i.e. Humanoids}
We've adressed many of the simple walking models in previous chapters. It turns out that there at some point by adding more and more DoFs, the OC gets easier again. This is the case if we can nearly arbitrarily place our foots wherever we want.

For Humanoids new tools come along, which will be covered within this chapter. 


\section{Center of Mass Dynamics}
When we think of a robots base with massless legs, it's actually comparable to a Hovercraft flying around in space. We have to add some constraints: 
\begin{itemize}
\item Constraining the input torque
\item The thrusters are movable
\item Thrusters are off when they are moving
\item Velocity is limited
\item Thrusters only can be turned on in certain regions
\end{itemize} 
Therefore the analogy makes sense: Masslegs legs are comparable to little thrusters.

One strategy that many humanoids use is to
\begin{itemize}
\item Keep the center of mass at a constant height and then
\item Minimize the angular momentum about the center of mass
\end{itemize}
This is a totally different view then before, since it is on a much higher level. And it turns out it is even meaningful if we consider legs that do have masses. 

Then, if we are really able to stabilize the CoM and total angle of momentum, the problem formulation gets a way easier, since
\begin{itemize}
\item we can effectively neglect the impulse dynamics, since the velocity of a foot touching the ground will be nearly zero!
\item The EoMs are linear!
\end{itemize}
 

\section{The Special Case of Flat Terrain}
This is the environment where most humanoids start walking. In this situation the constraints on the center of mass dynamics can be summarized very efficiently.

Let us define the "center of pressure" (CoP) as the point on the ground where 
$$x_{cop} = \frac{\sum_i p_{i,x} F_{i,z}}{\sum_i
    F_{i,z}},$$
and since all $p_{i,z}$ are equal on flat terrain, $z_{cop}$ is just the height of the terrain. It turns out that the center of pressure is a "zero-moment point" (ZMP) -- a property that we will demonstrate below -- and moment-balance equation gives us a very important relationship between the location of the CoP and the dynamics of the CoM: 
$$(m\ddot{z} + mg)
    (x_{cop} - x) = (z_{cop} - z) m\ddot{x} - I\ddot\theta.$$
    
If we use the strategy proposed above for ignoring collision dynamics, $\ddot{z}=\ddot{\theta}=0$, then we have $z-z{cop}$ is a constant height $h$, and the result is the famous "ZMP equations": 
$$\ddot{x} = -\frac{g}{h}
    (x_{cop}-x).$$
So the location of the center of pressure completely determines the acceleration of the center of mass, and vice versa!


\section{Zero Moment Point (ZMP) Planning}
For rigid body systems I can always summarize the contributions from many external forces as a single wrench (force and torque) on the object. For walking robots, it is this point on the ground from which the external wrench can be described by a single force vector (and no moment) that is the famous "zero-moment point".

On the spatial case of flat terrain $x{cop}$ is a ZMP!

And we can use this simple relation to derive more complicated plans for the robot. State of the art is a three step walking plan/control:
\begin{enumerate}
\item Footstep planning (Decide where to place the foots)
\item Plan COM \& COP
\item Fill in the details ($q,\dot{q}$)
\end{enumerate}  
Some remarks/insights on this algorithm:
\begin{itemize}
\item CoP always needs to be within the support polygon
\item It also works for pure position control -> Based on the footplan you derive an exact CoP trajectory that is a set of straight lines and follow it - Timing is important
\item From a control perspective, walking fast is easier than walking slowly. This is because the CoM is more straight while walking slowly the CoM trajectory is closer to the Footsteps (i.e. more shifting)
\item Earlier: people hardcoded the CoP trajectory
\item Now: Since the equations appeared are linear, we can use linear optimal control to solve for the CoP/CoM trajectory! E.g. you can optimize the robots behaviour to follow a desired CoM trajectory as close as possible while reducing the amount of energy
\end{itemize}


\section{Footstep-Planning and Push Recovery}
There was a time when footstep-planning where thought of an AI problem, but we can do better now. Some methods for coming up with footstep plans include
\begin{itemize}
\item Sampling-based
\item Nonlinear-trajectory optimization
\item Mixed-integer convex optimization
\end{itemize}
The special thing about footstep plans is that they use very simple models (if any models at all). Especially using
\begin{itemize}
\item Simplified kinematics and 
\item Almost no dynamics (e.g. inverted-pendulum models)
\end{itemize}
during the general planning. 

\textbf{Clarification: Static vs. Dynamic Walking:}
\begin{itemize}
\item With quasi static walking, the CoM at each point in time has to be within the support polygon of the robot (low efficiency, low $v_{max}$)
\item Dynamic Walking in contrast is not constrained as hard in this way. It allows for 'instable' intermediate phases. Nevertheless, the CoP has to be within the support polygon.  
\end{itemize}
Asimo uses static walking, ZMP is in this sense unstable but it is an desired pattern towards more efficient and faster locomotion. 


\section{Whole-Body Control}


\section{Beyond ZMP Planning}






