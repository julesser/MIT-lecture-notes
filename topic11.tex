\chapter{Lecture 18: Feedback Motion Planning}
\textbf{[Complete when online notes have been added!]} 

The randomized motion planning methods from the last chapter deliver impressive results. However, their application to complex robots is limited due to efficiency and robustness issues of just \textbf{random movements} with your robot in order to get to the road. 

In other words: Solely thinking about resulting \textbf{trajectories} probably wont win the game. This is especially true since complex interactions with the environment have a highly need of \textbf{providing feedback} from the done actions and consider them for the next control steps. 

\section{Probabilistic Feedback Coverage}
In order to provide robustness for complex tasks, one can think of the following approach
\begin{itemize}
\item We imagine a 'funnel' (robustness area) around our desired trajectory
\item This funnel is formed via some Lyapunov function (cubic spline)
\item A complex robots behavior (e.g. walking and then trotting) can be modeled as phases where different 'skills' are required
\item Each skill (i.e. controller), bounded by a Lyapunov function, ensures that it's last is feasible for the init of the following up skill 
\end{itemize}

\section{Drawbacks}
\begin{itemize}
\item Offline control method
\item Again, this method highly depends on the knowing the whole state space of the robot. 
\item Therefore, it also does not suit very well for complex systems
\end{itemize}