\chapter{Lecture 17: Motion Planning as Search}
Another tool for our toolbox: Complete Planning is one of the key problems left for us to tackle. This especially means to always find an answer to the question: Is it possible for this robot to reach a specific goal or not.

Two state of the art approaches include
\begin{itemize}
\item Decompose non-convexity in problem, then search all decompositions
\item Randomized motion problem
\end{itemize}  

The \textbf{future solution} might lay somewhere in between: While the randomized algorithms are extremely powerful, it somehow feels a bit strange that we do not make (partly) use of the geometric knowledge of the environment that we have from perception.


\section{Artificial Intelligence as Search}
Def: A planner is said to be \textbf{complete}, if it is guaranteed to find a solution if one exists.

Def: A planner is said to be \textbf{globally optimal} if it finds the optimal plan.

\section{Randomized Motion Planning}
These algorithms are fairly simple, extremely powerful and became state of the art nowadays. They belong to the group of \textbf{sampling-based planners}. The RRT is one algorithm that can be applied to underactuated systems. 
\subsection{Rapidly-Exploring Random Trees (RRTs)}
The basic algorithm: 
\begin{enumerate}
\item Generate a random sample
\item Discard the ones initialized within boundaries of objects
\item Find closest point on current track
\item Extend tree towards sample point
\end{enumerate}
\subsection{RRTs for Robots with Dynamics}
It shows that RRTs even work surprisingly well for high-dimensional search spaces, i.e. robots with many DoFs. If you know the kinematics well. 


\subsection{Variations and Extensions}
Possible improvements are
\begin{itemize}
\item Better sampling distributions
\item Better distance metric
\item Better extension 
\end{itemize}


\section{Decomposition Methods}
The idea:
\begin{itemize}
\item Decompose the environment into accessible regions, i.e regions where the quadrotor should stay within
\item Search for all of these decompositions
\end{itemize}
This approach is an option, but it is an \textbf{very expensive} way for seeking towards completeness. 